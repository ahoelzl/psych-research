  \chapter{Introduction}
   \chapter{Methods}
   \section{hierarchial methods}
   hierarchial clustering with average, complete and correlation, correlation of correlation
   \section{k-means}
   introduction of novel clustering method
   \section{number of clusters}
   introduction of shilouette and APN methods
   \chapter{Results}
   \section{traditional simulation study}
   \subsection{number of cluster}
   Problem: man bekommt keine Verteilung der Clusteranzahlen wie bei Stichprobendaten sondern immer nur genau eine Anzahl pro Korrelationsmatrix -> wie darstellen und auswerden
   \subsection{cluster simuliarity}
   create artificial EFA data with changing crossloading, factor correlations and determine which 
   cluster method is best for finding the structure of the cluster
   \section{real world simulation study}
   \subsection{number of clusters}
   distribution of number of clusters for different sample sizes. A method for determining the number of clusters is good if the number of clusters is stable and close to the number of clusters for the whole data set 
   \subsection{cluster similiarity}
   cluster similiarity between sample data and whole data with fixed cluster size. the closer the better. Oder dynamische Clustergröße?
   \section{cross validation with CFA}
with the one/two best methods for determining the number of clusters as well as the best cluster methods from the two chapters before 

\chapter{conclustion}
new cluster method that does very well in the known and newly invented cluster quality criteria. 