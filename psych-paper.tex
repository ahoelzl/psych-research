
\documentclass[bibtotoc,liststotoc,BCOR5mm,DIV12, headings=openany, oneside]{scrbook}
\usepackage[left=2cm,right=3cm,top=2cm,bottom=2cm,includeheadfoot]{geometry}
\usepackage{bibgerm}
\usepackage[utf8]{inputenc}
\usepackage{graphicx}
\usepackage{url}
\usepackage[colorlinks, linkcolor=black, urlcolor=blue,citecolor=black ]{hyperref}
\usepackage[ngerman]{babel}
\usepackage{tabularx}
\usepackage{animate}
\usepackage{amsmath}
\usepackage{listings}
\usepackage{amssymb,amsmath}
\usepackage{xcolor}
\usepackage{float}
\usepackage{transparent}
\usepackage{setspace}
\usepackage{subfig}
\usepackage{amsthm}
%\usepackage[round]{natbib}
\usepackage{multirow}
\usepackage{framed} % für farbige Umrahmungen von Definitionen
\definecolor{shadecolor}{rgb}{0.9,0.9,0.98} % legt Farbe der Kästen fest
\usepackage{float}
\restylefloat{figure}

\onehalfspacing
% für Tabellen mit farbigen Zellen
\usepackage{colortbl}
%\usepackage[round]{natbib}

%für Definitionen etc, die nicht automatisch kursiv sind:
\usepackage{amsthm}
%für Definitionen in farbigen Kästen:
\usepackage{shadethm}

% Für schönen R-Code:%%%%%%%%%%%%%%%%%%
\usepackage{color}
\usepackage{listings,relsize}
\usepackage{textcomp}
\lstloadlanguages{R}
\lstdefinelanguage{Renhanced}[]{R}{%
  morekeywords={acf,ar,arima,arima.sim,colMeans,colSums,is.na,is.null,%
    mapply,ms,na.rm,nlmin,replicate,row.names,rowMeans,rowSums,seasonal,%
    sys.time,system.time,ts.plot,which.max,which.min, quartz, pdf},
  deletekeywords={c},
  alsoletter={.\%},%
  alsoother={:_\$}}
\lstset{language=Renhanced,extendedchars=true,
  basicstyle=\small\ttfamily,
  commentstyle=\textsl,
  keywordstyle=\mdseries,
  showstringspaces=false,
  index=[1][keywords], 
  upquote=true, 
  indexstyle=\indexfonction,
  keywordstyle=\color[rgb]{0,0,1},
  commentstyle=\color[rgb]{0.133,0.545,0.133},
  stringstyle=\color[rgb]{0.627,0.126,0.941},
  literate=%
{~} {$\sim$}{1}
{Ö}{{\"O}}1
{Ä}{{\"A}}1
{Ü}{{\"U}}1
{ß}{{\ss}}2
{ü}{{\"u}}1
{ä}{{\"a}}1
{ö}{{\"o}}1
}
\newcommand{\indexfonction}[1]{\index{#1@\texttt{#1}}}
%\newcommand{\R}[1]{\lstinline{#1}}
\newcommand{\grWidth}{8cm}
%%%%%%%%%%%%%%%%%%%%%%%
\theoremstyle{definition}
\newtheorem{definition}{Definition}[section]
\newtheorem{example}{Beispiel}[section]
% Nummerierung regeln:
\numberwithin{equation}{section} %sets equation numbers <chapter>.<section>.<index>
\numberwithin{definition}{section} %sets equation numbers <chapter>.<section>.<index>

% Zeichen festlegen:
% Lower Prevision:
\newcommand{\LP}{
\underline{P}
}
%Upper Prevision:
\newcommand{\UP}{
\bar{P}
}
%Lower natural Extension
\newcommand{\LNE}{
\underline{E}
}
%Upper natural Extension
\newcommand{\UNE}{
\bar{E}
}

\newcommand{\opt}{
\text{opt}
}

\newcommand{\maxim}{
\text{opt}_{>_{\LP}}
}

\newcommand{\ivd}{
\sqsupset_{\LP}
}

\newcommand{\sdom}{
>_{\LP}
}

\newcommand{\ext}{
\text{ext}
}

\newcommand{\norm}{
\text{norm}
}

\newcommand{\nfd}{
\text{nfd}
}

\newcommand{\st}{
\text{st}
}

\newcommand{\back}{
\text{back}
}

\newcommand{\gamb}{
\text{gamb}
}
% Für "Italien"
\newcommand{\I}{
\shorthandoff{"}I\shorthandon{"}
}


%Entscheidungsknoten:
\newcommand{\EK}{
D
}

%Zufallsknoten:
\newcommand{\ZK}{
C
}

%Entscheidung:
\newcommand{\En}{
d
}

%Ereignis
\newcommand{\Er}{
E
}

\newcommand{\rew}{
u
}

\newcommand{\ev}{
\text{ev}
}
% Komischen Einzug verhindern:
\setlength{\parindent}{0em}

%\numberwithin{equation}{subsection} %sets equation numbers <chapter>.<section>.<subsection>.<index>
%\numberwithin{equation}

%\begin{document}{subsubsection} %sets equation numbers <chapter>.<section>.<subsection>.<subsubsection>.<index>
% wahrscheinlich auch mit chapter möglich.


\begin{document}


 
    %\input{/home/andreas/Desktop/Zusammen/InputBA/title.tex} 
    %\thispagestyle{empty}
 % \clearpage
   % \input{/home/andreas/Desktop/Zusammen/InputBA/Abstract.tex}
  %  \thispagestyle{empty}




\frontmatter 
 
    \thispagestyle{empty}
    \tableofcontents 
% ---------------------------------------------------------------
\mainmatter % die eigentliche Arbeit

  \chapter{Introduction}
   \chapter{Methods}
   \section{hierarchial methods}
   hierarchial clustering with average, complete and correlation, correlation of correlation
   \section{k-means}
   introduction of novel clustering method
   \section{number of clusters}
   introduction of shilouette and APN methods
   \chapter{Results}
   \section{traditional simulation study}
   \subsection{number of cluster}
   Problem: man bekommt keine Verteilung der Clusteranzahlen wie bei Stichprobendaten sondern immer nur genau eine Anzahl pro Korrelationsmatrix -> wie darstellen und auswerden
   \subsection{cluster simuliarity}
   create artificial EFA data with changing crossloading, factor correlations and determine which 
   cluster method is best for finding the structure of the cluster
   \section{real world simulation study}
   \subsection{number of clusters}
   distribution of number of clusters for different sample sizes. A method for determining the number of clusters is good if the number of clusters is stable and close to the number of clusters for the whole data set 
   \subsection{cluster similiarity}
   cluster similiarity between sample data and whole data with fixed cluster size. the closer the better. Oder dynamische Clustergröße?
   \section{cross validation with CFA}
with the one/two best methods for determining the number of clusters as well as the best cluster methods from the two chapters before 

\chapter{conclustion}
new cluster method that does very well in the known and newly invented cluster quality criteria. 


% ---------------------------------------------------------------
%\backmatter % ab hier keine Nummerierung mehr


 %   \listoffigures % Abbildungsverzeichnis 

 %   \bibliographystyle{apalikeGer}
 %   \bibliography{BALit3}

\chapter{ }

\end{document}
